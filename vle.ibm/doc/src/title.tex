\newcommand{\reporttitle}{Une extension du logiciel GVLE}     % Titre
\newcommand{\reportsubtitle}{Modélisation individu centré}     % SousTitre
\newcommand{\reportauthor}{Geneviève \textsc{Cirera} (SI4)} % Auteur
\newcommand{\reportsubject}{Rapport de Stage 4ème année} % Sujet
\newcommand{\HRule}{\rule{\linewidth}{0.5mm}}
\setlength{\parskip}{1ex} % Espace entre les paragraphes

\begin{titlepage}

\begin{center}

\begin{minipage}[t]{0.49\textwidth}
\vspace*{-2cm}
  \begin{flushleft}
    \includegraphics [width=55mm]{images/LogoUNSA.jpg} \\[0.6cm]
    
  \end{flushleft}
\end{minipage} 
\begin{minipage}[t]{0.49\textwidth}
  \begin{flushright}
    \includegraphics [width=55mm]{images/Logo_polytech_SI.jpg} \\[0.2cm]
  \end{flushright}
\end{minipage} \\[2.0cm]
\vspace*{-1cm}
\textsc{\Large Ingénieur en Sciences Informatiques}\\[0.5cm]
\textsc{\Large \reportsubject}\\[0.5cm]
\HRule \\[0.4cm]
{\huge \bfseries \reporttitle}\\[0.4cm]
{\Large \bfseries \reportsubtitle}\\[0.2cm]
\HRule \\[1.5cm]
\vspace*{-1cm}
\begin{minipage}[t]{0.64\textwidth}
  \begin{flushleft} \large
    \emph{Stagiaire :}\\
    \reportauthor
  \end{flushleft}
\end{minipage}
\begin{minipage}[t]{0.35\textwidth}
  \begin{flushleft} \large
   % \emph{Tuteur :} \\
    %M.~Frédéric \textsc{Precioso} \\[0.5cm]
    \emph{Maître de stage :} \\
    Patrick \textsc{Chabrier}
  \end{flushleft}
  \vspace*{0.8cm}
\end{minipage}
%\vspace*{7.0cm}
\textsc{\Large Entreprise INRA (Toulouse)}\\[0.5cm]
{\emph{16 juin 2014 - 20 septembre 2014}}\\
\vspace*{0.8cm}
\textsc{\Large Résumé}\\[0.5cm]
\justify
Ce stage s'est déroulé dans l'équipe-projet RECORD au sein de l'Institut National de la Recherche en Agronomie de Toulouse, un organisme français de recherche fondé en 1946.\\
A partir d'un logiciel existant, j'ai travaillé de manière autonome tout au long du stage afin de développer une extension logicielle sous la forme de plugin. Ce plugin développé en C++ a pour but de faciliter la modélisation d'individus au niveau de troupeau.\\
Après s'être familiarisé avec le logiciel, plusieurs aspects ont été abordés, définition des nouveaux besoins, développement des fonctionnalités, de l'interface graphique, debogage, présentation des résultats...
\vfill

%{\large 5 mars 2012}
\center
\vspace*{0.2cm}
\includegraphics [width=45mm]{images/logoINRA.jpg} \\[0.6cm]
\end{center}

\end{titlepage}
